\documentclass[conference]{IEEEtran}
\IEEEoverridecommandlockouts
% The preceding line is only needed to identify funding in the first footnote. If that is unneeded, please comment it out.
\usepackage{cite}
\usepackage{amsmath,amssymb,amsfonts}
\usepackage{algorithmic}
\usepackage{graphicx}
\usepackage{textcomp}
\def\BibTeX{{\rm B\kern-.05em{\sc i\kern-.025em b}\kern-.08em
    T\kern-.1667em\lower.7ex\hbox{E}\kern-.125emX}}
\begin{document}

\title{A Research on Semantic Middleware\\

}

\author{\IEEEauthorblockN{Shanliang Yao}
\IEEEauthorblockA{\textit{Department of Computer Science and Software Engineering} \\
\textit{Xi’an Jiaotong-Liverpool University}\\
Suzhou, China \\
Shanliang.Yao19@student.xjtlu.edu.cn}
}

\maketitle

\begin{abstract}
This document is a model and instructions for \LaTeX.
This and the IEEEtran.cls file define the components of your paper [title, text, heads, etc.]. *CRITICAL: Do Not Use Symbols, Special Characters, Footnotes, 
or Math in Paper Title or Abstract.
\end{abstract}

\begin{IEEEkeywords}
component, formatting, style, styling, insert
\end{IEEEkeywords}

\section{Introduction}
\subsection{Background}

At the beginning of the World Wide Web, the content on the network was only human readable, but the computer could not understand and process it. For example, when we browse a webpage, we can easily understand the content on the webpage, and the computer only knows that this is a webpage. There are pictures and links on the webpage, but the computer does not know what the pictures are about, and it is unclear how the link points to the current page. The Semantic Web is a general framework proposed to make data on the network machine-readable. "Semantic" is to express the meaning behind the data in a richer way, so that the machine can understand the data. "Web" hopes that these data link with each other to form a huge information network, just like the interconnected web pages in the Internet, but the basic unit becomes smaller data.

Linked data was originally used to define how to use the Semantic Web technology to publish data on the Internet, which emphasizes the creation of links between different data sets. Linked data, as a technology for publishing data on the World Wide Web, has achieved good results. Linked data uses Resource Description Framework (RDF) as the data model. It is a set of technical specifications of markup languages proposed by the World Wide Web Consortium (W3C). It is based on the XML grammar and XML Schema data types in order to describe and express the content and structure of network resources more abundantly. Linked data uses HTTP as a protocol, similar to how HTML, HTTP, and URLs are used to build a Web. The powerful distributed open graphical data model of Linked Data makes it ideal for integrating data stored in various databases and file systems, as well as integrating applications related to such data.

Middleware is a collection of common parts in the integration process for information system interaction. It shields complex and general functions such as communication, interaction, and connection at the bottom, and is provided in the form of products. When the system is interacting, middleware is used to connect and interact directly, avoiding a lot of code development and labor costs. With the development of the Internet, the pressure on the server is increasing, and the distributed framework is widely used in the project. Through the middleware, each business system can be separated and bear the pressure alone. Such an architecture ensures the high availability of the project. The emergence of semantic middleware has helped computers better understand the content of the network and greatly promoted the development of the semantic web.

\subsection{Research Question}
The development of semantic middleware is getting faster in the technical field. The use of semantic middleware in practical projects is also of great significance. To gain a deeper understanding of semantic middleware and its function, my research questions are described as follows: 
How to apply semantic middleware to real-world applications and what benefits it can bring to this area.

\section{Literature Review}

There are many types of middleware software that can be used to solve various problems for either enterprise or platform purposes. In this research, I mainly review three kinds of middleware, namely Dubbo, RFID and Seata, and explain their advantages and disadvantages with some technical details.

\subsection{Dubbo}

Dubbo is a high-performance, lightweight service middleware open sourced by Alibaba and can be seamlessly integrated with the Spring framework. It provides three core capabilities: interface-oriented remote method invocation, intelligent fault tolerance and load balancing, and automatic service registration and discovery.

\subsubsection{Categorization}

Dubbo is a type of Remote Procedure Call (RPC) middleware which is a process where a client application calls a remote server. Its basic idea is to keep the syntax of the client and server programs the same, as if they were on the same machine. The application calls the remote procedure just like the local procedure, starts the operation of the remote procedure, and then returns the result of the operation to the local procedure.

\subsubsection{Strengths}

\paragraph{High-performance RPC call for interface proxy}
Provides high-performance agent-based remote calling capabilities, services with interface as the granularity, shielding developers from the underlying details of remote calls.

\paragraph{Intelligent load balancing}
Built-in multiple load balancing strategies to intelligently sense the health status of downstream nodes, significantly reduce call latency and improve system throughput.

\paragraph{Service automatic registration and discovery}
Support multiple registration center services, real-time perception of service instances online and offline.

\paragraph{Highly scalable}
Following the design principle of microkernel + plug-in, all core capabilities such as Protocol, Transport, and Serialization are designed as extension points, treating built-in implementations and third-party implementations equally.

\paragraph{Traffic scheduling during operation}
Built-in routing strategies such as conditions and scripts. By configuring different routing rules, it is easy to realize gray-scale publishing and priority in the same room.

\paragraph{Visual service governance and operation and maintenance}
Provide rich service management and operation and maintenance tools: query service metadata, service health status and call statistics at any time, issue routing strategies in real time, and adjust configuration parameters.


\subsubsection{Limitations}
Dubbo also has some limitations, such as:

\begin{itemize}
\item The language support is not friendly, only supports JAVA language.

\item The development of Dubbo is more difficult, because the dependency of dubbo's jar package can not be solved by many large projects.
\end{itemize}


\subsection{RFID}

The RFID middleware is located between the RFID system and the application system, and is responsible for data transmission between the RFID system and the application system.
It is responsible for solving the problems of reliability, security and data format conversion of RFID data.
The connection between the RFID middleware and the RFID system is implemented using the application program interface (API) provided by the RFID system. After the data in the RFID card is read by the reader, it is transmitted to the RFID middleware through the API program. After the RFID middleware processes the data, it publishes the data externally through standard interfaces and services.

\subsubsection{Categorization}

RFID middleware is a kind of Message-Oriented Middleware (MOM). Information is transmitted in the form of message from one program to another program or multiple programs. MOM contains functions not only for passing information, but also for interpreting data, security, data broadcasting, error recovery, locating network resources, finding paths that meet costs, prioritizing messages and requirements, and other services.

\subsubsection{Strengths}

\paragraph{Insulation Infrastructure}
RFID middleware is independent and interposed between RFID readers and back-end applications, and can be connected to multiple RFID readers and multiple back-end applications to reduce the complexity of architecture and maintenance.

\paragraph{Data Flow}
The main purpose of RFID is to convert physical objects into virtual objects in the information environment, so data processing is the most important function of RFID. RFID middleware has the characteristics of data collection, filtering, integration and transmission, so as to pass the correct object information to the enterprise's back-end application system.

\paragraph{Process Flow}
RFID middleware uses program logic and store-and-forward functions to provide sequential message flows, and has the ability to design and manage data flows.

\paragraph{Standard}
RFID is an application of automatic data sampling technology and identification of physical objects. The Electronic Product Code (EPC) is stored in the RFID, and after being read by the RFID reader, it can provide the tracking of the item name and related information represented by the EPC, and immediately identify and share the item data in the supply chain, effectively providing information transparency .

\subsubsection{Limitations}
 At present, the main suppliers of RFID middleware platforms are IBM, Oracle, Microsoft, SAP, and Sun. For these manufacturers, RFID middleware is just an extension of their existing software, and their RFID products can be quickly and easily integrated with their respective existing software product lines. But the disadvantage is that its products depend on other vendors' software products.
 
 
 \subsection{Seata}
\subsubsection{Categorization}
Seata is an Alibaba open source distributed Transaction Processing Middleware (TPM), in order to provide high-performance and easy-to-use distributed transaction services under the microservice architecture.

Seata's design idea is that a distributed transaction can be understood as a global transaction. A branch transaction is a local transaction that satisfies atomicity, consistency, isolation, and durability (ACID). Therefore, we can operate distributed transactions just like local transactions.


\subsubsection{Strengths}
\paragraph{No impact on business}
The impact here refers to the design and transformation of applications at the business level due to the constraints of the technical issue of distributed transactions. This design and modification often bring high research and development and maintenance costs to the application. Solving the distributed transaction problem at the middleware level does not require the application to do extra work at the business level.

\paragraph{High performance}
High performance: Introducing the guarantee of distributed transactions, there will inevitably be additional overhead, causing performance degradation. Reduce the performance loss introduced by distributed transactions to a very low level, so that applications will not be affected by the availability of services due to the introduction of distributed transactions.

\subsubsection{Limitations}
Seata can only be used in relational databases that support ACID.
The transaction isolation level supports up to the level of read submission, and the analysis of SQL cannot cover all the syntax.

\section{Methodology}

I have performed a research on the real-world applications and systems in different areas that make use of the semantic middleware.Next, I enumerate three industries in which semantic middleware plays an important role.

\subsection{E-commerce system}

Micro-services are widely used in e-commerce systems. It advocates the division of monolithic applications into a set of several small services. The services coordinate and cooperate with each other to provide users with e-commerce services.

\subsubsection{Use of middleware}
Dubbo is Alibaba's open source distributed service framework and is widely used by various Internet companies. The maturity of Dubbo middleware itself and the completeness of documents can basically meet the business needs of various Internet companies. Distributed application scenarios have high concurrency, high scalability and high performance requirements. It also involves serialization/deserialization, networking, multithreading, and design patterns. Fortunately, the Dubbo framework encapsulates the above knowledge, allowing programmers to focus on the business.

\subsubsection{Benefits of middleware}
Dubbo middleware solves complex business scenarios, rapid business growth, and endless stability issues. Dubbo supports all trading and non-transactional business systems of Alibaba Group (Taobao, Tmall), and safely passed the trading challenge of 11.11 shopping festival.
Many other e-commerce platforms also use Dubbo to help their systems run smoothly and provide users with services.

\subsection{Internet of Things}
The proposition of the semantic concept of the Internet of Things stems from the need to solve the heterogeneous interconnection and platform intelligence formed during the rapid development of the Internet of Things. The IOT semantics describe resources in a semantic way, which makes the description of resources have good readability and is easy for machine to understand and process, so as to realize the functions of semantic operations, semantic query, semantic reasoning and semantic combination.

\subsubsection{Use of middleware}
Middleware is located between the underlying RFID hardware equipment (such as radio frequency identification reader) and the back-end database and application software (such as ERP system) in the RFID application system. It filters, summarizes and calculates the tags-related events and data from the reader, reducing the huge amount of raw data transmitted from the reader to enterprise applications

Relying on the powerful functions of RFID technology in identification, perception, networking, positioning, etc., applying it to the management of the manufacturing process of complex parts can enhance the intelligent image of products and improve the quality and service of product design, production, sales, after-sales, and maintenance. Level, to achieve the full life cycle of the product, and effectively improve its manufacturing efficiency and quality.

\subsubsection{Benefits of middleware}
For application developers, the important function of RFID middleware lies in the powerful event processing and software management mechanism unique to the product. The event processing engine helps developers easily establish, deploy, and manage an end-to-end logical RFID processing process, which is completely independent of the underlying specific device model and information exchange protocol between devices. Because the mode of logical devices is used in the event processing engine, the RFID data processing process can be truly separated from the physical topology of the devices that the application deployment phase faces, thus greatly reducing the complexity of the design and not having to care about the supply of these devices Quotient and what communication protocol is used between them.

RFID middleware can also be effectively combined with such as Enterprise Resource Planning (ERP) system, warehouse management system (WMS) and other proprietary business systems for business processing. This good adaptability makes the RFID application built using this framework only need to make a very small amount of program changes to work seamlessly with the original business system software.

\subsection{Finance}
The data consistency requirements of the financial industry are relatively high. That is to say, if a step fails, then either roll back to the previous service call, or keep retrying to ensure that all steps are successful. The business process under the microservices architecture in the financial field tends to be more complicated and long. For example, it is normal for an Internet micro-loan business process to adjust more than a dozen services, and the process of exception handling is more complicated.

\subsubsection{Use of middleware}
Seata defines three modules to handle the relationship and processing of global transactions and branch transactions. The three components are:

\paragraph{Transaction Coordinator (TC)}: A transaction coordinator that maintains the running state of global transactions and is responsible for coordinating and driving the submission or rollback of global transactions.
\paragraph{Transaction Manager (TM)}: controls the boundary of global transactions, is responsible for starting a global transaction, and finally initiates a resolution of global commit or global rollback.
\paragraph{Resource Manager (RM)}: controls branch transactions, is responsible for branch registration, status reporting, and receives instructions from the transaction coordinator to drive the submission and rollback of branch (local) transactions.

\subsubsection{Benefits of middleware}
The financial industry must do centralized processing. This is not simply a matter of adding or subtracting amounts. It involves changes in the back-end logical data such as customer accounts, household accounts, and general ledger accounts. All accounting systems must be managed in accordance with the rules. Borrowing and lending must be completed in real time in a logical processing transaction unit and guarantee ACID.

Seata has been used on Alipay for a long time, and maintains the stable operation of Alipay. When some servers are down, it can still switch other servers. And when the transaction is abnormal, it still maintains the consistency of the data. It is precisely because of seata's high reliability that its use in the financial field is getting wider and wider.


\section{Conclusion}

\subsection{Summary}



\subsection{Future Work}
After this research, I will continue to study the theory and technology of semantic middleware, and compare the functions and implementation principles of different middleware.

Besides, I will also deeply study its application in the Internet of Things, combine middleware and the Internet of Things, and do some innovative development work to better understand the meaning of middleware in software architecture.



\begin{table}[htbp]
\caption{Table Type Styles}
\begin{center}
\begin{tabular}{|c|c|c|c|}
\hline
\textbf{Table}&\multicolumn{3}{|c|}{\textbf{Table Column Head}} \\
\cline{2-4} 
\textbf{Head} & \textbf{\textit{Table column subhead}}& \textbf{\textit{Subhead}}& \textbf{\textit{Subhead}} \\
\hline
copy& More table copy$^{\mathrm{a}}$& &  \\
\hline
\multicolumn{4}{l}{$^{\mathrm{a}}$Sample of a Table footnote.}
\end{tabular}
\label{tab1}
\end{center}
\end{table}

% \begin{figure}[htbp]
% \centerline{\includegraphics{fig1.png}}
% \caption{Example of a figure caption.}
% \label{fig}
% \end{figure}


\section*{Acknowledgment}

The preferred spelling of the word ``acknowledgment'' in America is without 
an ``e'' after the ``g''. Avoid the stilted expression ``one of us (R. B. 
G.) thanks $\ldots$''. Instead, try ``R. B. G. thanks$\ldots$''. Put sponsor 
acknowledgments in the unnumbered footnote on the first page.

\section*{References}

Please number citations consecutively within brackets \cite{b1}. The 
sentence punctuation follows the bracket \cite{b2}. Refer simply to the reference 
number, as in \cite{b3}---do not use ``Ref. \cite{b3}'' or ``reference \cite{b3}'' except at 
the beginning of a sentence: ``Reference \cite{b3} was the first $\ldots$''

Number footnotes separately in superscripts. Place the actual footnote at 
the bottom of the column in which it was cited. Do not put footnotes in the 
abstract or reference list. Use letters for table footnotes.

Unless there are six authors or more give all authors' names; do not use 
``et al.''. Papers that have not been published, even if they have been 
submitted for publication, should be cited as ``unpublished'' \cite{b4}. Papers 
that have been accepted for publication should be cited as ``in press'' \cite{b5}. 
Capitalize only the first word in a paper title, except for proper nouns and 
element symbols.

For papers published in translation journals, please give the English 
citation first, followed by the original foreign-language citation \cite{b6}.

\begin{thebibliography}{00}
\bibitem{b1} Wikipedia contributors. "Resource Description Framework." Wikipedia, The Free Encyclopedia. Wikipedia, The Free Encyclopedia, 15 May. 2020. Web. 4 Jun. 2020.
\bibitem{b2} J. Clerk Maxwell, A Treatise on Electricity and Magnetism, 3rd ed., vol. 2. Oxford: Clarendon, 1892, pp.68--73.
\bibitem{b3} I. S. Jacobs and C. P. Bean, ``Fine particles, thin films and exchange anisotropy,'' in Magnetism, vol. III, G. T. Rado and H. Suhl, Eds. New York: Academic, 1963, pp. 271--350.
\bibitem{b4} K. Elissa, ``Title of paper if known,'' unpublished.
\bibitem{b5} R. Nicole, ``Title of paper with only first word capitalized,'' J. Name Stand. Abbrev., in press.
\bibitem{b6} Y. Yorozu, M. Hirano, K. Oka, and Y. Tagawa, ``Electron spectroscopy studies on magneto-optical media and plastic substrate interface,'' IEEE Transl. J. Magn. Japan, vol. 2, pp. 740--741, August 1987 [Digests 9th Annual Conf. Magnetics Japan, p. 301, 1982].
\bibitem{b7} M. Young, The Technical Writer's Handbook. Mill Valley, CA: University Science, 1989.
\end{thebibliography}

\end{document}
